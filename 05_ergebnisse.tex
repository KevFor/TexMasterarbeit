\chapter{Ergebnisse}
\section{Massenspektren bzw Strom}
hier Spektren bzw Strom in Chtr bzw FC2 zeigen. Optiken funktionieren, anschauen von Fokus von Fe1 im FC2.
Für Strahlfokus die entsprechenden Abbildungen zeigen.
\section{Strahlfokus}
So wie in Vortrag zeigen, dass hier nur FC1 Strom
Hier Fokus \ref{fig:faltung} statt unter Kapitel Simulation
\begin{figure}
    \centering
    \includegraphics[scale=0.9]{./fig/Faltung2.pdf}
    \caption{Abbildung überarbeiten und genau beschreiben oben mitte unten jewiels gneau.}
    \label{fig:faltung}
\end{figure}
\section{Probleme? schwankungen, strom läuft weg - oxidieren}
Stabilität des Stroms
Schwanken des Stroms, vor und nach Depo bzw zwischendurch kontrollieren - Strom sinkt bzw schiebt zu anderen Massen















\newpage
\section{UV/vis Spektroskopie}
Die UV/vis Spektroskopie wird im Rahmen dieser Arbeit genutzt um den Depositionsfleck zu finden.
Dafür wurde der.....
Ablauf \ref{fig:uvvis_raster} vlt in anderem/früheren Kapitel

\begin{figure}
    \centering
    \includegraphics[scale=0.6]{./fig/uvvis_raster-probe.png}
    \caption{l. raster, oben r. Glasträger mit ITO / Probe ITO in sample holder bild raus
     zeilenweise gerastert sagen dass x8 y15 mehrmals aufgenommen wurde, refrenzstelle für optische stabilität    Caption Caption Caption Caption Caption Caption Caption .}
    \label{fig:uvvis_raster}
\end{figure}

\subsection{Optische Stabilität}
Wie in Kapitel \ref{sec:scoutsim} beschrieben, sollten die Eisen Cluster eine Absorption von weniger als ein Prozent bewirken.
Ein Störeffekt der bei der UV/vis Spektroskopie auftreten kann ist die Veränderung der Intensität der Lampe. 
% Dies führt dazu, dass auf der gleichen Probenstelle eine veränderte Absorption gemessen wird.
Daher wurde bei dem Rastern der Probe nach jeder Zeile die Referenzstelle wiederholt gemessen.
In Abbildung \ref{fig:uvvis_ref} ist diese Veränderung innerhalb einer Messreihe zu sehen.
Das erste und letzte Spektrum liegen dabei etwa eine Stunde außeinander und die Lampe wurde ca. drei Stunden vor Beginn der Messreihe eingeschaltet.
\begin{figure}
    \centering
    \includegraphics[scale=0.8]{./fig/uvvis_ref.png}
    \caption{Abbildungs schöner machen.. achsenbeschriftung label etc., y-Skala zu Prozent machen Caption überarbeiten
    Absorptionsmessungen einer Referenzstelle im Laufe einer Messreihe.
    Die Spektren entsprechen den jeweiligen Zeilen im gleichen Farbton aus Abb. \ref{fig:uvvis_raster}.}
    \label{fig:uvvis_ref}
\end{figure}
Die Intensität der Lampe ist im Laufe der Messung größer geworden und übersteigt die der Referenzmessung $I_Ref$. 
Dies führt zu einer erhöhten Transmission, bzw. einer negativen Absorption.

beschreiben steigen von I null
größenordnung und dmait Messgenauigkeit hinsichtlich Lampe







Referenz im Laufe der Messung zeigen.
Keine Cluster im Rahmen der Messgenauigkeit zu sehen.

\subsection{Probeninhomogenität}

Hier Heatmap und Kratzer zeigen
Messgenauigkeit hinsichtlich Probeninhomogenität
Inhomegintät eine Prozent so wie Signal
Kratzer deutlich stärker, mehrere Prozent / deutlich mehr als 1 Prozent in Absorption

\begin{figure}
    \centering
    \includegraphics[scale=0.5]{./fig/uvvis_map.png}
    \caption{Caption Caption Caption Caption Caption Caption Caption Caption Caption Caption Caption Caption Caption Caption Caption Caption Caption Caption .}
    \label{fig:uvvis_map}
\end{figure}

\subsection{Probleme Inbetriebnahme}
Mehr Strom/Depodauer nötig (Diskussion), gestaltet sich aber schwierig das umzusetzen.. oxidieren justage strom haut ab blabla
\section{AFM}
Ito lässt sich abbilden, Bild dazu, Aussage über Rauhigkeit treffen, genügend glatt?? \


\subsection{ITO Transmission}

Verschiedene Anzahl Schichten und ggf verschieben

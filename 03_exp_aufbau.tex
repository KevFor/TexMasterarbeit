\chapter{Experimenteller Aufbau}
Für die Untersuchung und Erzeugung von Clustern gibt es einen Aufbau der sich in zwei Komponenten unterteilen lässt, vgl. Abb \ref{fig:aufbau}.
Im Rahmen dieser Arbeit waren das Präparations-/Analysesystem und die Clusterstrahlanlage (kurz CSA) voneinander getrennt.
Hinter der CSA wird anstelle des bisherigen Aufbaus die Schleusen Kammer in Betrieb genommen.
% Mit dieser soll die Erzeugung und schnelle Entnahme von Proben, für optische Messungen, realisiert werden.
% CSA ruhig kurz halten und auf Dominik und Yin verweisen, ggf andere die irgendwas speziell gemacht haben
\begin{figure}
    \centering
    \includegraphics[scale=0.23]{./fig/aufbau.png}
    \caption{Experimenteller Aufbau mit der bisher genutzten Präparationskammer und den drei Abschnitten der Clusterstrahlanlage, \cite[S. 20]{wolter} adaptiert. 
    In grün ist der Verlauf des Clusterstrahls eingezeichnet.}
    \label{fig:aufbau}
\end{figure}
\section{Clusterstrahlanlage}
Die Clusterstrahlanlage kann in drei Abschnitte unterteilt werden, die Clusterquelle, die Kryo Kammer und den Massenselektor, vgl. Abb \ref{fig:aufbau}.
In diesen werden die Cluster erzeugt, der Clusterstrahl fokussiert und die gewünschte Clustergröße eingestellt. Die genauere Funktionsweise wird im folgenden beschrieben
\subsection{Clusterquelle}
Prinzip der Erzeugung ist das.
\subsection{Kryo Kammer}
auf Diplomarbeit krause verweisen, gut erklärt
Fokussierung und Restgas weg.
\subsection{Massenselektor}
Gewünschte Clustergröße einstellen 
Massenselektor \cite{Issendorff.1999}
\section{Schleusen Kammer}
kurz beschreiben und auf nächste Kapitel hinweisen - Simulaton bzw Konstruktion

\section{UV/vis und AFM Aufbau erklären}
UV/vis und ggf AFM hier erklären? wohin, afm überhaupt?
Foto von UV/vis Aufbau und ggf hier Messablauf erklären

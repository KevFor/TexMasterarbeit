\chapter{Motivation}
Die Clusterphysik beschäftigt sich mit den einzigartigen Eigenschaften von Clustern (Ansammlung von zwei bis zu einigen tausend Atomen) welche weder durch die Atom-/Molekülphysik noch die Festkörperphysik vollständig beschrieben werden.
In diesem Übergangsbereich finden sich je nach Größe und Material verschiedene Eigenschaften.\\
Die Motivation dieser Arbeit liegt in der Untersuchung von massenselektierten magnetischen Clustern in Halbleiter-Quantum-Wells, mit dem langfristigen Gedanken das supermagnetische Limit in magnetischen Speichermedien zu überwinden.
Ferromagnet-Halbleiter-Hybridstrukturen, welche die funktionalen Eigenschaft beider Systeme vereinen sollen, sorgen bereits für großes Interesse.
Schichtsysteme aus ferromagnetischem Kobalt auf dem Halbleiter CdTe, getrennt durch eine (Cd, Mg)Te Barriere, wurden untersucht und zeigten eine weitreichende robuste Kopplung \cite{Korenev.2016}.
Durch die Deposition von massenselektierten magnetischen Clustern, wie z.B. Eisen, sollen zunächst die größenabhängigen magnetischen Eigenschaften untersucht wurden.
Für magneto-optische Untersuchungen bietet sich zunächst die Verwendung eines transparenten Substrats an, da 



Ein neuer Ansatz soll durch die Deposition von massenselektierten magnetischen Clustern in Halbleiter-Quantum-Wells Strukturen geliefert werden.
Im Fokus liegt die Betrachtung der magneto-optischen Eigenschaften von magnetischen Clustern, mit dem Ziel die Effekte der stark lokalisierten magnetischen Felder in den 


von stark lokalisierten magnetischen Felder, erzeugt durch magnetische Cluster beispielsweise aus Eisen, mit optischen Methoden.
Dafür wird 
Zunächst Deposition von großen Eisenclustern auf transparentem Halbleiter um



Magnetische Nanopartikel wie Eisen sind von
Ito-Fe-cluster composite films and Fe-doped-ITO films have interesting properties \cite{Peng.2005, Ohno.2007, Shen.2015}



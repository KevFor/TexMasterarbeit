\chapter{Diskussion und Ausblick}
% Zu Strahlfokus, bisher nur Fe$_1$ Fokus, sofern bei anderen Ionen besseres Signal-rausch-Verhältnis, dann auch hier Fokus anschauen.
\section{Inbetriebnahme}
Für die Schleusenkammer wurden neue Ionenoptiken konstruiert, simuliert und erfolgreich in Betrieb genommen.
Die Simulation zeigten auf, dass ohne Depositionsoptiken keine stabile Strahlführung möglich ist.
In Abschnitt \ref{sec:fokus1} ergab sich für den konvergenten bzw. divergenten Strahl auf dem Faraday cup ein Fokus mit weniger als $\SI{0,3}{\mm}$ bzw. $\SI{2}{\mm}$ Durchmesser.
Durch den Einbau von Depotube sowie Probenhalter+Blende war es möglich den Ionenstrahl im neuen Aufbau zu detektieren.
Für große Eisencluster, hier $Fe_{1790}$, ergab sich ein maximaler Clusterstrom von $\SI{2}{\pA}$ welche jedoch nicht über mehrere Stunden stabil blieb.\\
Mit dem Signal von $Fe_1$ wurde der Fokus vermessen und für das Strahlprofil, unter Annahme einer Gauß-Verteilung, eine Halbwertsbreite von $\SI{1,5}{\milli\meter}$ bestimmt.
Bei der Verwendung des Oberflächenanalyse-Systems und den entsprechenden Depositionsoptiken hatte der Depositionspot einen Durchmesser von etwa $\SI{1}{\mm}$ \cite[S. 40]{gronhagen}.
Der neue Aufbau bestehend aus Depotube und Blende ermöglicht somit eine angemessene Fokussierung des Clusterstrahls.\\

Für eine Verbesserung der Fokussierung könnte die zurzeit verwendete Blende durch eine längere zylinderförmige ersetzt werden, welche den Ionenstrahl bis kurz vor die Proben führt. 
Die damit verbundene höhere Symmetrie der elektrischen Felder sollte sich positiv auf die Stabilität der Strahlenbahn auswirken.
In Zukunft könnte die Schleusenkammer parallel zum bestehenden Präparations- und Analysesystem genutzt werden.
Dafür wäre ein Aufbau nötig, bei dem der Clusterstrahl durch Ablenkelektroden entweder in die Schleusenkammer oder in die Präparationskammer geleitet wird, ähnlich wie es bei dem Ausgang des Massenselektors mit dem Channeltron ist.
In diesem neuen Übergang bietet sich auch ein Plattenventil an, damit die Schleusenkammer zur schnellen Probenerzeugung und Entnahme genutzt werden kann.


% ....vergleich zu simulation...
\section{Große Eisencluster}
Mit großen Eisenclustern wurden erste Depositionsexperimente auf ITO durchgeführt und eine Probe mit einer Belegung von etwa $0,15$ Clustermonolagen erzeugt.
Durchgeführte Berechnungen für diese Depositionsmenge ergaben eine erwartete Absorption von etwa $0,3 - 0,6\,\%$, siehe Abschnitt \ref{sec:scoutsim}.
Mit dem optischen Aufbau können Subprozent Unterschiede in der Absorption aufgelöst werden und prinzipiell ist die Detektion des Depoflecks zu erwarten.
Die Untersuchung der Probe zeigte allerdings lokale Inhomogenitäten des Substrats auf und die damit verbundenen Abweichungen der Absorption sind etwa $\SI{1.5}{\%}$.
Des Weiteren wurde eine Absorption von $\SI{6.7}{\%}$ gemessen, welche sich auf einen Kratzer in makroskopischer Größenordnung zurückführen lässt, mit einem Lichtmikroskop sichtbar.
Die Charakterisierung des Depositionsflecks war daher bisher nicht möglich und eine Verbesserung der UV/vis Spektroskopie ist nötig.\\

Weil die Probe makroskopisch Inhomogenitäten aufweist, die größer sind als das zu erwartende Signal, ist hier eine Verbesserung erforderlich.
Durch die Präparation der Substrats sollten sich die Absorptionsfehler verringern.
Ergänzend dazu bietet sich die Vermessung des Substrats vor und nach der Deposition an, um so eine bestehende Inhomogenität des Substrats heraus zu rechnen.
Die Stabilität des Aufbaus und eine hohe Reproduzierbarkeit muss dabei gewährleistet sein, da sonst die Messung der leeren Probe hinfällig ist.
Eine weitere Verbesserung könnte sich durch die Realisierung eines \textit{in situ} UV/vis Spektroskopie Aufbaus ergeben, weil dadurch die Probenposition stabil bleibt.
Verbunden mit den Vakuumbedingungen wäre auch eine geringere Kontamination der Probe.
Sofern eine Implementierung der UV/vis Spektroskopie in den knappen Abmessungen der Schleusenkammer möglich ist, besteht weiterhin das Problem von zeitlichen Effekten.
Wie in Kapitel \ref{sec:uvvis_stabilitaet} deutlich wurde, verändert sich die Intensität der Lampe bereits in einen Zeitraum von $\SI{90}{\min}$, dieser Effekt ist bei Depositionsdauern von mehreren Stunden nicht zu vernachlässigen.\\

Für die zukünftig geplanten magnetooptischen Untersuchungen von Halbleiter-Quantum-Well Strukturen sollten, nach der erfolgreichen Charakterisierung des Depositionsflecks auf ITO, andere Substrate verwendet werden.
Im ersten Schritt bietet sich die Erzeugung und Untersuchung von Proben mit magnetischen Clustern auf nichtmagnetischen Halbleitersubstraten, wie CdTe oder GaAs, an.


%\chapter{Ausblick}

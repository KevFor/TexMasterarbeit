\chapter{Motivation}
Die Clusterphysik beschäftigt sich mit den Eigenschaften von Clustern (Ansammlung von wenigen bis zu einigen tausend Atomen) welche weder durch die Atom-/Molekülphysik noch die Festkörperphysik vollständig beschrieben werden.
In diesem Übergangsbereich finden sich je nach Größe, Form und Material einzigartige strukturelle, elektronische, magnetische und optische Eigenschaften \cite{Jena.1992}, so zeigen z.B. Dichtefunktionaltheorie Berechnungen auf, dass Magnetismus für die Stabilität von kleinen Eisenclustern eine große Rolle spielt \cite{Kim.2014}.\\
Ferromagnet-Halbleiter-Hybridstrukturen, welche die funktionalen Eigenschaften beider Systeme vereinen sollen, sorgen seit geraumer Zeit für großes Interesse \cite{Ohno.1998, Zakharchenya.2005, Dietl.2010}.
Schichtsysteme aus ferromagnetischem Kobalt auf dem Halbleiter CdTe, getrennt durch eine (Cd, Mg)Te Barriere, wurden untersucht und zeigten eine weitreichende robuste Kopplung auf \cite{Korenev.2016}.\\
Die Motivation dieser Arbeit liegt in der Untersuchung von massenselektierten magnetischen Clustern in Halbleiter-Quantum-Wells, mit dem langfristigen Ziel das supermagnetische Limit in magnetischen Speichermedien zu überwinden.
Durch die Deposition von massenselektierten magnetischen Clustern, wie z.B. Eisen, sollen zunächst die magnetischen Eigenschaften untersucht werden.\\
Dafür ist im ersten Schritt die Deposition auf Indiumzinnoxid (engl. inidum tin oxide, kurz ITO) geplant, da dieses Substrat optische Messungen in Transmission erlaubt.
Verbundsysteme aus ITO und Fe Clustern weisen selber interessante optische, elektrische und magnetische Eigenschaften auf und sind daher durch die Kombination aus Ferromagnetismus und Halbleiter von Interesse für die Spintronik \cite{Peng.2005, Ohno.2007, Shen.2015}.\\
Im Fokus liegt zunächst die ex-situ Betrachtung der magnetooptischen Eigenschaften von magnetischen Clustern, deshalb wird im Rahmen dieser Arbeit an die bestehende Anlage zur Clustererzeugung ein neuer Aufbau in Betrieb genommen und erste Proben mit großen Eisenclustern sollen hergestellt werden.
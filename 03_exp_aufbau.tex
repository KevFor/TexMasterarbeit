\chapter{Experimenteller Aufbau}
Für die Untersuchung und Erzeugung von Clustern gibt es einen Aufbau welcher sich in zwei Komponenten unterteilen lässt, vgl. Abb \ref{fig:csaaufbau}.
Im Rahmen dieser Arbeit waren das Präparations-/Analysesystem und die Clusterstrahlanlage (kurz CSA) voneinander getrennt, mehr Informationen zum bisherigen Aufbau finden sich beispielsweise in \cite{wolter}.
An den letzten Abschnitt der CSA wird anstelle der Präparationskammer die Schleusen Kammer platziertz und in Betrieb genommen.
% Mit dieser soll die Erzeugung und schnelle Entnahme von Proben, für optische Messungen, realisiert werden.
% CSA ruhig kurz halten und auf Dominik und Yin verweisen, ggf andere die irgendwas speziell gemacht haben
\begin{figure}
    \centering
    \includegraphics[width=\textwidth]{./fig/aufbau.png}
    \caption{Experimenteller Aufbau mit der bisher genutzten Präparationskammer und den drei Abschnitten der Clusterstrahlanlage, \cite[S. 20]{wolter} adaptiert. 
    In grün ist der Verlauf des Clusterstrahls eingezeichnet.}
    \label{fig:csaaufbau}
\end{figure}

\section{Clusterstrahlanlage}
Die Clusterstrahlanlage kann in drei Abschnitte unterteilt werden, die Clusterquelle, die Kryo Kammer und den Massenselektor, vgl. Abb \ref{fig:csaaufbau}.
In diesen werden die Cluster erzeugt, der Clusterstrahl fokussiert und die gewünschte Clustergröße eingestellt.
Die genauere Funktionsweise der einzelnen Elemente wird im folgenden beschrieben, mehr Information finden sich z.B. in den Arbeiten von \cite{duffe, schröder, wolter}.
\subsection{Clusterquelle}
Für die Erzeugung der Cluster wird eine Magnetron-Sputter-Gas-Aggregations-Quelle benutzt, bei der aus einem Target Atome gesputtert werden, welche dann in einer Edelgasatmosphäre zu Clustern kondensieren, siehe Abb. \ref{fig:quelle}.
Das Target ist auf einem Magnetron Kopf montiert vgl. Abb \ref{fig:magnet}, durch Einlassen eines Argon-Helium Gemisches und Anlegen einer Spannung von etwa $\SI{1}{\kV}$ wird ein Plasma gezündet, vgl. Abb. \ref{fig:plasma}
Verwendet wird ein $\SI{1}{\mm}$ dünnes Eisen Target welches eine Einkerbung besitzt um den Sputterprozess zu begünstigen, siehe Abb. \ref{fig:target1}, \ref{fig:target2}.
Ohne Einkerbung und bei einem dickeren Target würde das Magnetfeld zu stark abgeschwächt werden.
Die gesputterten Atome können in dem Aggregationsbereich, Raum zwischen Target und Iris, zu Clustern kondensieren, wobei bis zu $80\,\%$ der entstandenen Cluster ioniesiert werden, \cite{Haberland.1991}.
Es ensteht ein breites Spektrum an Clustern dessen Verteilung kann durch verschiedene Parameter varieert werden.
Diese sind die Menge und Zusammensetzung des Gas-Gemisches, die Leistung mit der gesputtert wird, Größe des Aggregationsbereichs und der variablen Irisöffnung, sowie die Potentiale der Ionenoptiken in der Quelle.\\

Mit einem Phywe Teslameter (Auflösung $\SI{0,01}{\milli\tesla}$) und einer tangentialen Hallsonde, aus der Vorlesungsvorbereitung der Fakultät Physik, wurde das Magnetfeld vermessen.
% (Serial No: 190400147191) (Serial No: 130400145524)
Gemessen wurde die senkrechte B-Feldkomponente am Rand und in der Mitte des Targets sowie die parallele B-Feldkomponente über der Einkerbung, siehe Abb. \ref{fig:quelle}. 

\begin{table}
    \centering
    \caption{Magnetfeld der Quelle mit und ohne Target, gemessen wurde die senkrechte Komponente am Rand und in der Mitte sowie die parallele Komponente über der Einkerbung.}
    \label{tab:bfeld}
    \begin{tabular}{l|lllll}
        \toprule
        ohne Target	&	Mitte	&	oben	&	unten	&  links & rechts\\
        B$_{\bot}$ / mT   & 450 &-140 & -115& -130& -130 \\
        B$_{\parallel}$ / mT   & 0 &130 & 130& 130& 130 \\    
        \midrule
        mit Target	&	Mitte	&	oben	&	unten	&  links & rechts\\
        B$_{\bot}$ / mT   & 44 &-11 & -23& -18& -20 \\
        B$_{\parallel}$ / mT   & 0 &36 & 34& 34& 32 \\      
       \bottomrule
    \end{tabular}
  \end{table}
  

\begin{figure}
    \centering
    \includegraphics[width=\textwidth]{./fig/quelle.png}
    \caption{Schematischer Aufbau der Clusterquelle, \cite{woltermaster}.}
    \label{fig:quelle}
\end{figure}


\begin{figure}
    \centering
    \begin{subfigure}[h]{0.34\textwidth}
        \includegraphics[width=\textwidth]{./fig/YinMagnet1.png}
        \caption{}
        \label{fig:magnet}
    \end{subfigure}\hfill
    \begin{subfigure}[h]{0.34\textwidth}
        \includegraphics[width=\textwidth]{./fig/YinMagnet2.png}
        \caption{}
        \label{fig:plasma}
    \end{subfigure}\hfill
    \begin{subfigure}[h]{0.31\textwidth}
        \includegraphics[width=\textwidth]{./fig/Fetarget.png}
        \caption{}
        \label{fig:target1}
    \end{subfigure}\hfill
    \begin{subfigure}[h]{1\textwidth}
        \includegraphics[width=\textwidth]{./fig/Fetarget2.png}
        \caption{}
        \label{fig:target2}
    \end{subfigure}
    \caption{Oben: Links Magnetkonfiguration, mitte Plasmaentladung \cite{Yin.2007} und rechts eingbautes Eisen Target. Unten: Eisen Target ohne und mit Einkerbung, altes benutztes Target.}
    \label{fig:target}
\end{figure}


\subsection{Kryo Kammer}
Nach der Quelle gelangen die Cluster in die Kryokammer, hier wird der diffuse Strahlanteil mit einem Skimmer abgeschnitten und mit Ionenoptiken der Strahl fokussiert sowie auf $\SI{500}{\eV}$ beschleunigt, siehe Abb. \ref{fig:kryokammer}.
Der darauf folgende X/Y-Deflektor ermöglicht eine Verkippung des Strahls senkrecht zur Strahlrichtung.
Für die Bereinigung des Ionenstrahls vom Restgas wird mit einer Kryopumpe eine Temperatur von etwa $\SI{10}{\kelvin}$ \cite[S. 27]{woltermaster} an der ersten Kryostufe erzeugt, dadurch kann das austretende Argon ausgefroren werden.
Weitere Informationen zu den einzelnen Komponenten finden sich in \cite{krause}.
\begin{figure}
    \centering
    \includegraphics[width=\textwidth]{./fig/kryokammer.png}
    \caption{Schematischer Aufbau der Kryokammer, \cite{woltermaster}.}
    \label{fig:kryokammer}
\end{figure}

\subsection{Massenselektor}
In dem Massenselektor, entwickelt von R. Palmer und B. von Issendorff\cite{Issendorff.1999}, wird der Ionenstrahl durch Nutztung des Time-of-Flight Prinzips aufgespalten, siehe Abb. \ref{fig:mschtr}.
Durch eien zeitlich begrenzten Beschleunigungspuls wird ein Ionenpaket senkrecht zur Strahlrichtung abgelenkt und alle Cluster erhalten die gleiche Energie, wobei kleinere Cluster wegen ihrer geringeren Masse eine höhere Geschwindigkeit haben.
Nachdem das Ionenpaket den Driftraum passiert wird ein Abbremspuls angelegt und die Cluster sind aufgrund der unterschiedlichen Geschwindigkeiten der Masse nach separiert.
Große Cluster befinden sich weiter oben und kleinere weiter unten, dadurch werden mit einem kleinen Austrittsloch nur Cluster der eingestellten Masse durchgelassen.
Im Massenselektor kann der Ionenstrom aller Cluster mit der Messelektrode gemessen werden, sofern keine Pulse angelegt werden.
Mithilfe von zwei Ablenkelektroden wird der massenselektrierte Ionenstrahl in ein Channeltron geleitet um dort ein verstärkten Strom zu messen.
Ohne das Anlegen der Ablenkspannungen an den Elektroden werden die Cluster weiter in den nächsten Abschnitt geleitet.

\begin{figure}
    \centering
    \includegraphics[width=\textwidth]{./fig/massenselektor_3.png}
    \caption{Hier Bild MS, insbersondere Chtr und elektrode zum umschalten.}
    \label{fig:mschtr}
\end{figure}

\section{Schleusen Kammer}
\label{sec:schleuse}
Hinter dem Massenselektor befindet sich ein Übergang bestehend aus einer verschiebbaren Ionenoptik, der \textit{movable tube} kurz M-Tube, welche in einem Wellbulk befestigt ist, siehe Abb. \ref{fig:depotube}.
Darauf folgt ein Plattenventil und bisher die Präparationskammer mit den dazugehörigen Depositionsoptiken.
In dieser Arbeit wird stattdesssen die Schleusenkammer angebracht, siehe Abb. \ref{fig:schleuse}.
Mit dieser soll die Erzeugung und schnelle Entnahme von Proben für \textit{ex situ} optische Messungen realisiert werden.
In dieser war zunächst lediglich ein Faraday cup montiert und die Lineardurchführungen sind noch ungenutzt.
Für die Inbetriebnahme werden neue Depositionsoptiken benötigt welche in Kapitel \ref{sec:inbetriebnahme} beschrieben werden.

\begin{figure}
    \centering
    \includegraphics[width=\textwidth]{./fig/schleuse_5.png}
    \caption{Schleusenkammer abgekoppelt vom restilichen Aufbau, über Plattenventil und Wellbulk verbunden mit dem Massenselektor, vgl Abb. \ref{fig:depotube}. Im inneren befindet sich ein Faraday cup sowie zwei Lineardruchführungen.}
    \label{fig:schleuse}
\end{figure}


\section{UV/vis Spektroskopie Aufbau}
Für die UV/vis Spektroskopie wird ein Aufbau genutzt, wie er in Abb. \ref{fig:uv_visaufbau} zu sehen ist.
Das Licht einer Halogen-Deuterium Lampe, im Wellenlängenbereich von etwa $(200-1200)\,$nm, wird mit einer Kollimator Linse auf die Probe fokussiert.
In einer weiteren Linse wird das Licht auf einem Glasfaserkabel eingefangen und an ein Spektrometer geleitet.
Durch eine Blende mit $\SI{1}{\milli\meter}$, $\SI{0,6}{\milli\meter}$ und $\SI{0,3}{\milli\meter}$ wird die örtliche Auflösung gewährleistet und die Probe ist auf einem in drei Dimensionen verfahrbaren Präzisionshalter montiert.\\
Vor Beginn einer Messreihe wird ein Dunkelspektrum bei ausgeschalteter Lichtquelle aufgenommen und vom gemessenen Signal abgezogen. 
Mit dem Signal $I_{Ref}$ einer Referenzstelle und dem Signal $I$ einer beliebigen Probenstelle ergibt sich die Absorption 
\begin{align}
    % T=\left(\frac{I-I_{dark}}{I_{Ref}-I_{dark}}\right)\\
    A=- \lg\left(T\right) = -\lg\left(\frac{I-I_{dark}}{I_{Ref}-I_{dark}}\right),
\end{align}
wobei T die Transmission und $I_{Dark}$ das Signal bei ausgeschalteter Lampe ist.

\begin{figure}
    \centering
    \includegraphics[width=\textwidth]{./fig/uv_visaufbau_6.png}
    \caption{Die Komponenten des Messaufbaus für die UV/vis Spektroskopie sind montiert auf einer optischen Bank, ebenfalls abgebildet sind die dazugehörige Halogen-Deuterium Lampe und das Spektrometer.}
    \label{fig:uv_visaufbau}
\end{figure}

\chapter{Diskussion und Ausblick}
% Zu Strahlfokus, bisher nur Fe$_1$ Fokus, sofern bei anderen Ionen besseres Signal-rausch-Verhältnis, dann auch hier Fokus anschauen.

optische Messungen verbessern

Leerrasterung und dann gleiche STellen nach Depo - Problem Tage/Wochen abstand zwischen Leermessung und nach Depo



% Für die Untersuchung von großen Eisen Clustern wird eine Belegung von $0,1-0,2$ Monolagen angestrebt, weil dann die relalive Zahl koaleszierter Cluster weniger als 40 \% betragen sollte\cite[S. 88]{bsieben}.
% Für das Profil des Ionenstrahls wird eine Gauss Verteilung erwartet, daher sollten auch bei höheren Belegungen die Randbereiche des Depositionsflecks genügend separierte Cluster aufweisen.
% Für das Auffinden des Depospots bietet sich eine höhere Belegung von einer Monolage an, weil die erwartete Absorption dann auch größer ist.

%\chapter{Ausblick}

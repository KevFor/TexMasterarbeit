% \chapter{Anhang}
% Eigentlich gehören in den Anhang Materialien die der Argumentation der Arbeit dienen, z.\,B. technische Zeichnungen oder Quellcodes.
% Hier soll der Anhang jedoch anders genutzt werden.
% In diesem Kapitel sollen einige Dinge aufgeführt und kurz beschrieben werden die möglicherweise von Interesse sein könnten.
% Manches findet sich sowieso in der Arbeit, den Laborbüchern oder aber es ist versteckt in Fotos oder in den Köpfen von uns.
% Es besteht keinerlei Garantie auf Vollständigkeit aber es wird versucht einige Dinge festzuhalten.
% Außerdem werden die Formulierungen deutlich umgangssprachlicher und der Inhalt unstruktierter sein, trotzdem viel Spaß mit den Infos!

% FC Messungen mit e gun
% e gun von Luci plaga funktioniert
% Fehler 6
% Bilder der CSA
% Konstruktionszeichnung und registernummer
% Ablagerung auf Hülse - Gaszufuhr bewegliches rohr nicht ganz drin
% Quelle zegt deutlich weniger abhängigkeit von Heliumfluss
% Mischcluster Spekren 2018-10-29_01_mass150-500_wait500_step1_tw40_Chtr_Quelle1,2cm_Iris8,68
% Schwankung Kryo / periodizität etwa 2.4s spektren 20180625
% B-feld messung der Magnetron Kopfes
% temp Messung mit thermoelement hinten an quelle


\chapter{Theoretische Grundlagen}
In diesem Kapitel....
cluster - \cite{haberland}
uv vis \cite{kreibig}

Was soll hier rein? ITO transmission bzw mit fe? scout sachen?

\section{Cluster erlären}
Haberland
\section{ITO hier?}
Warum Ito, wie sieht ITO aus
was gibts für tolle Sachen  bei ITO
Fe in auf Ito bzw Ito modifiziert
ITO erfüllt Anforderungsprofil für Depo/Untersuchung
Hopg, MoS2 (atomar glatt, inert - AFM, einfach präparierbar - abziehen)
Glasträger mit ITO beshcichtet Sigma aldrich, technische Daten wie schichtdicke
\section{Depozeit}
\label{sec:depo}
sagen dass sich clusterstrom bei einfach geladenen teilchen in x ionen übersetzen lässt
dann das bei Strahlgröße xmm sich ein Depofleck ergibt der so viele cluster benötig werden um eine ML zu bilden
damit sagen dass x pAmin 1 ML sind
das entspricht x min bei y pA oder 10x min bei 0.1 y pA


% Deposition Heinz: 0.1 ev/atom sanft 1ev/atom okay 10/ev Atom nicht saft - Bindungsenergie
Abschätzung wie viel kann deponiert werden vgl Ag / Cu Zahlen nennen
Rechnung wie viel Belegung gewollt 1/10 ML
The most severe restriciton for clusters on supports conecerns the taotal coverage which must remain well below one cluster monolayer in order to avoid coalescence processes.
This might (although it need to) be a drawback in optical experiments.
It can be overcome y evaporating on mylar foils and subsequently folding them together [3.76 (Referenz in Buch)].
Kreibig \cite[S.216]{Kreibig.1995} S.216 - optical properties

Text gegebenfalls mit Formeln im Text \\
Ar$^*$ + Fe$_n$ $\rightarrow $ Ar + Fe$^+_{n}$ + e$^-$ \\
und hier nur Formeln in align Umgebung\\
\begin{align}
  N(l)=&\frac{10l^3 -15l^2 + 11l -3}{3} \\
  \text{Clusterstrom}&\approx 2\,\text{pA} \quad \widehat{=} \quad \frac{2\cdot10^{-12}\,\dfrac{C}{s}}{1,6\cdot10^{-19}\,C} = 1,25\cdot10^7 \,\dfrac{\text{Ionen}}{s} \\
  A_{\text{Depofleck}}&=\pi\cdot(\frac{1,5}{2}\cdot10^{-3}\,\text{m})^2 \\
  A_{\text{Cluster}}&=\pi\cdot(\frac{3,5}{2}\cdot10^{-9}\,\text{m})^2 \\
  \frac{A_{\text{Depofleck}}}{A_{\text{Cluster}}}&= \text{Cluster je Monolage}\approx 4,6\cdot10^{11} \,\text{Ionen}\\
  t&=\frac{\text{Clusterstrom}}{\text{Cluster je Monolage}}\approx 4\cdot10^4\,\text{s} \approx 11\,\text{h} \\
  n_{\text{Monolayer}}&=\frac{A_{\text{Spot}}}{A_{\text{Cluster}}}
\end{align}
\blindtext \\
Für Poster \\
$Fe_{100\text{k}\,\text{amu}} \quad\approx\quad Fe_{1790} $
\begin{align*}
 \text{\#Cluster per Monolayer}\\
 N_{ML}=\frac{\text{Spotsize$\,\cdot\,$packing density}}{\text{Clustersize}} \approx  4,6\cdot10^{11}\,\frac{\text{Ions}}{\text{ML}} \\
 \text{Deposited $\approx 200\,$pAmin}\quad \widehat{=} \quad  7,5\cdot10^{10} \,\text{Ions} \\
 \frac{\text{Deposited}}{\text{\#Cluster per Monolayer}}\quad \rightarrow \quad 0.15\,\,\text{ML}
\end{align*}
\begin{align}
 \text{Cluster Current}&\approx 1\,\text{pA} \quad \widehat{=} \quad 6,24\cdot10^6 \,\dfrac{\text{Ions}}{s} \\
 \text{Deposition time}&\approx 200\,\text{min} 
\end{align}

\section{UV/vis und AFM bisschen Theorie?}
dunkelmessung
signal eine stelle - Ref
andere stelle signal
signal durch ref und bei beiden dark abziehen


Ref Dark - I dark I ref und I messung
\begin{align*}
    T=\left(\frac{I-I_{dark}}{I_{Ref}-I_{dark}}\right)\\
    A=- \lg\left(T\right) = -\lg\left(\frac{I-I_{dark}}{I_{Ref}-I_{dark}}\right)
\end{align*}


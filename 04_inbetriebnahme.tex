\chapter{Inbetriebnahme}
In diesem Kapitel wird zunächst die Konstruktion und Simulation der neuen Ionenoptiken in der Schleuse beschrieben.
Anhand des stabilen Signals von Fe$_1$ wurde der Fokus des Clusterstrahls in der Schleuse untersucht.
Danach sind die Massenspektren für kleine und große Cluster aufgeführt und es werden einige Störeffekte aufgeführt die beim Betrieb der CSA auftreten.
\section{Konstruktion}
Übergang - Depotube und Depooptiken halter/blende

Ionenoptiken in der Schleuse \ref{fig:schleuse_freecad} 

\begin{figure}
    \begin{subfigure}[t]{0.475\textwidth}
      \includegraphics[width=\textwidth]{./fig/schleuse_freecad.png}
      \caption{Konstruktion der Ionenoptiken in der Schleuse..., erzeugt mit freecad, cite freecad sofern möglich.}
      \label{fig:schleuse_freecad}
    \end{subfigure}\hfill
    \begin{subfigure}[t]{0.475\textwidth}
      \includegraphics[width=\textwidth]{./fig/schleuse_real.png}
      \caption{Ionenoptiken in der Schleuse eingbaut...}
      \label{fig:schleuse_real}
    \end{subfigure}
    \caption{\ref{fig:schleuse_freecad} Konstruktion und \ref{fig:schleuse_real} eingebaut.} 
    \label{fig:schleuse_innen}
\end{figure}


\section{Simulation}
Hier bisschen über simion \cite{Manura.2008} und yin \cite{Yin.2007} reden

Ionenoptiken in der Schleuse \ref{fig:simion_schleuse} 

\begin{figure}
    \centering
    \includegraphics[scale=0.5]{./fig/simion_schleuse.png}
    \caption{Simulation des Clusterstrahls in der Schleuse....}
    \label{fig:simion_schleuse}
\end{figure}
konv und div strahl, die entsprechenden spannungen einfügen, siehe poster bilder ordner bzw simion bilder namen

\begin{figure}
    \begin{subfigure}[h]{1\textwidth}
      \includegraphics[width=\textwidth]{./fig/simion_1.png}
      \caption{Konstruktion der Ionenoptiken in der Schleuse..., erzeugt mit freecad, cite freecad sofern möglich.}
      \label{fig:simion_1}
    \end{subfigure}\hfill
    \begin{subfigure}[t]{1\textwidth}
      \includegraphics[width=\textwidth]{./fig/simion_2.png}
      \caption{Ionenoptiken in der Schleuse eingbaut...}
      \label{fig:simion_2}
    \end{subfigure}
    \begin{subfigure}[h]{1\textwidth}
        \includegraphics[width=\textwidth]{./fig/simion_3.png}
        \caption{Konstruktion der Ionenoptiken in der Schleuse..., erzeugt mit freecad, cite freecad sofern möglich.}
        \label{fig:simion_3}
    \end{subfigure}\hfill
    \begin{subfigure}[t]{1\textwidth}
        \includegraphics[width=\textwidth]{./fig/simion_4.png}
        \caption{Ionenoptiken in der Schleuse eingbaut...}
        \label{fig:simion_4}
    \end{subfigure}
    \caption{Fokussierung des Ionenstrahl ausgehend von konvergentem und divergentem strahl.} 
    \label{fig:simion_param}
\end{figure}

\section{Strahlfokus}
Mit dem stabilen und starken Signal von Fe$_1$ wurde der Fokus des Ionenstrahls in der Schleuse untersucht.
Für große Cluster war dies wegen des schlechten Signal-Rausch-Verhältnisses nicht möglich.
Der Faraday cup wurde entlang einer Achse senkrecht zum Strahl verschoben und der Ionenstrom gemessen, siehe Abb. \ref{fig:faltung}.
Der gemessene Strom ergibt sich aus der Faltung des Faraday cups und der echten Strahlform.
Daher wurde für den Faraday Cup mit einer Öffnung von etwa $\SI{2.5}{\mm}$ und den Ionenstrahl eine Faltung durchgeführt und diese manuell an die Messdaten angepasst.
Bei der für den Clusterstrahl angenommenen Gauß-Verteilung ergibt sich für die Halbwertsbreite etwa $\SI{1.5}{\mm}$. 
Bei Betrieb der CSA mit dem Oberflächenanalyse System und den dazugehörigen Depositionsoptiken ist der Durchmesser des Depositionsspot in etwa $\SI{1}{\mm}$ \cite[S. 40]{gronhagen}.
% Verglichen damit kann mit dem simplen Aufbau der Ionenoptiken in der Schleuse ein angemessener Fokus erzielt werden.

\begin{figure}
    \centering
    \includegraphics[scale=0.9]{./fig/faltung2.pdf}
    \caption{Abbildung überarbeiten und genau beschreiben oben mitte unten jewiels gneau.}
    \label{fig:faltung}
\end{figure}


\section{Massenspektren bzw Strom}

Nur FC oder auch Chtr? \ref{fig:cluster_chtr}
Massenspektren kleine und große Cluster
Strom kann auch 2 pA sein, siehe Quelltext, name des Spektrums %2018-11-21_04_mass1000-250000_wait500_step1000_tw15_FC2_Quelle0cm_Iris9,08_13Watt
Strom 4 pA für Fe1 siehe FC messung in Abh von x
hier Spektren bzw Strom in Chtr bzw FC2 zeigen. Optiken funktionieren, anschauen von Fokus von Fe1 im FC2.
Für Strahlfokus die entsprechenden Abbildungen zeigen.
\ref{fig:cluster_chtr}

\begin{figure}
    \begin{subfigure}[t]{0.475\textwidth}
      \includegraphics[width=\textwidth]{./fig/cluster_small.pdf}
      \caption{Caption....}
      \label{fig:cluster_small}
    \end{subfigure}\hfill
    \begin{subfigure}[t]{0.475\textwidth}
      \includegraphics[width=\textwidth]{./fig/cluster_medium.pdf}
      \caption{Caption..}
      \label{fig:cluster_medium}
    \end{subfigure}
    \caption{\ref{fig:cluster_small} und \ref{fig:cluster_medium} Clsuterstrom in chtr für kleiner Cluster bis $Fe_{30}$.} 
    \label{fig:cluster_chtr}
\end{figure}

% \begin{figure}
%   \begin{subfigure}[h]{1\textwidth}
%     \includegraphics[width=\textwidth]{./fig/cluster_small.pdf}
%     \caption{Eisen Ionen mit 1-10 Atomen bei einer Channeltronspannung von HV $= \SI{1000}{\volt}$.}
%     \label{fig:cluster_small}
%   \end{subfigure}\hfill
%   \begin{subfigure}[t]{1\textwidth}
%       \includegraphics[width=\textwidth]{./fig/cluster_medium.pdf}
%       \caption{Clusterstrom gemessen mit dem Channeltron bei HV $= \SI{1000}{\volt}$. \SI{1500}{\volt}.}
%       \label{fig:cluster_medium}
%   \end{subfigure}
%   \caption{Clusterstrom gemessen mit dem Channeltron im Massenselektor. Quelle optimiert für kleinere Cluster mit bis zu 10 \ref{fig:cluster_small} bzw. 28 Atomen \ref{fig:cluster_medium}.} 
%   \label{fig:cluster_chtr}
% \end{figure}


\section{Störeffekte}
Für die Probenherstellung beziehungsweise die Untersuchung von Clustern ist ein hoher stabiler Clusterstrom essentiell.
Eine Abschätzung für die Depositionszeit gab es in \ref{sec:depo} und ein Strom von etwa $\SI{2}{\pA}$ konnte in der Schleuse für Fe$_{1790}$ realisiert werden.
Bei dem Betrieb der CSA gibt es allerdings unerwünschte Störeffekte wie Intensitätseinbrüche \cite[S. 45]{gust}.
Im Rahmen dieser Arbeit wurden Störeffekte bei der Messung des Stroms aber auch bei dem Betrieb der Clusterquelle beobachtet.\\

Zum einen ....

übliche werte irawan von ag ? siehe \cite - Eisen eine größenordnung strom weniger (Heinz), waurm? 


kurze zeitskala kryo schwanken stört optimieren
kryo aus option aber Untergrund steigt bzw generell mögliche verunreinigungen
Lnagzeit stabilität, mehrere stunden depodauer, strom sollte auf die se skala stabil sein
problem thermischer natur siehe strom in abh von leistung
und Abnahme des stroms oxidieren / nach justieren erforderlich
Stabilität des Stroms
Schwanken des Stroms (Kryo), vor und nach Depo bzw zwischendurch kontrollieren - 
Strom sinkt bzw schiebt zu anderen Massen (thermisch bzw Optiken-Iris, Skimmer beschichtet)

Mehr Strom/Depodauer nötig (Diskussion), gestaltet sich aber schwierig das umzusetzen.. 
oxidieren justage strom haut ab blabla


\chapter{Große Eisencluster auf ITO}
% Einige Schwierigkeiten bei der Inbetriebnahme bzw bei längerer Inbetriebnahme
Mit der erfolgreichen Inbetriebnahme der Schleusenkammer ist die Deposition auf ITO möglich. In desem Kapitel werden die Depositionsexperimente beschrieben.
% und die Depositionsmenge wird abgeschätzt.
Darauf folgen die Ergebnisse \textit{ex situ} UV/vis Spektroskopie Messungen. 

% auch AFM Messungen aber nur Substrat, hier Cluster nicht abgebildet

\section{Deposition}
\label{sec:depoauswertung}
% Für die Untersuchung von großen Eisen Clustern wird eine Belegung von $0,1-0,2$ Monolagen angestrebt, weil dann die relalive Zahl koaleszierter Cluster weniger als 40 \% betragen sollte\cite[S. 88]{bsieben}.
% Für das Profil des Ionenstrahls wird eine Gauss Verteilung erwartet, daher sollten auch bei höheren Belegungen die Randbereiche des Depositionsflecks genügend separierte Cluster aufweisen.
% Für das Auffinden des Depospots bietet sich eine höhere Belegung von einer Monolage an, weil die erwartete Absorption dann auch größer ist.
In Kapitel \ref{sec:depo} wurde berechnet, dass eine Depositionsmenge von etwa $\SI{1200}{\pA\min}$ einer Monolage an Clustern entspricht.
Da es zunächst darum ging erste Proben mit großen Eisen Clustern herzustellen und den Depositionsfleck zu charakterisieren wurde eine Belegung von etwa einer Monolage angestrebt. 
Das Maximum des Stroms war bei diversen Optimierungsversuchen zwischen $(80-100)\cdot 10^3\,\text{u}$.
Es wurde beobachtet, dass sich nach Start der Quellenbetriebs auch ohne Veränderung von Parametern das Maximum ein wenig zu höheren Massen verschiebt, wie es zum Beispiel auch bei der Erhöhung der Leistung \ref{sec:leistung} auftrat.
Daher wurde eine Masse von $\SI{100e3}{\amu}$, was $Fe_{1790}$ entspricht, für die Deposition auf ITO gewählt.\\

Durch Optimieren der Parameter der CSA konnte ein Strom von $\SI{2}{\pA}$ erreicht werden.
Es wurde 8 Stunden lang deponiert und danach der Ionenstrom erneut gemessen, siehe Abb. \ref{fig:firstdepo}.
Bei der ausgewählten Clustermasse war unter Berücksichtigung des Untergrunds kein Clusterstrom mehr vorhanden. 
Für den noch sichtbaren geringen Ionenstrom befand sich das Maximum der Verteilung bei etwa $\SI{160e3}{\amu}$.
Wegen diesen Tatsachen ist keine zuverlässige Abschätzung der Belegung möglich.\\
\begin{figure}[h]
  \begin{subfigure}[h]{\textwidth}
    \includegraphics[width=0.8\textwidth]{./fig/firstdepo.pdf}
    \caption{}
    \label{fig:firstdepo}
  \end{subfigure}\hfill
  \begin{subfigure}[t]{\textwidth}
    \includegraphics[width=0.8\textwidth]{./fig/seconddepo.pdf}
    \caption{}
    \label{fig:seconddepo}
  \end{subfigure}
  \caption{Ionenstrom von großen Clustern gemessen in der Schleuse bei ausgeschalter Kryopumpe. (a) Erstes Depositionsexperiment, abgebildet sind die Spektren vor und nach der Deposition, in der Zwischenzeit wurden keine Spektren gemessen. (b) Bei dem zweiten Depositionsexperiment wurde der Strom in regelmäßigen Abständen gemessen und gegebenfalls nachjustiert. Abgebildet ist die erste von zwei Depositionsetappen, weil nach einer gewissen Zeit keine Optimierung möglich war, wurde an einem anderen Tag erneut deponiert.} 
  \label{fig:deposition}
\end{figure}

In einem zweiten Depositionsexperiment wird der Ionenstrom alle $60 - \SI{90}{\min}$ kontrolliert und wenn möglich optimiert.
In Abb. \ref{fig:seconddepo} sind die Massenspektren vor, zwischen und nach einem Depositionsschritt abgebildet. 
Weil der Strom nach einem Depositionsschritt bei $\SI{0.6}{\pA}$ war und sich nicht optimieren lies, sondern soger weiter sank, wurde an einem anderen Tag erneut deponiert.
In Tabelle \ref{tab:depo2} ist eine Abschätzung für die Depositionsmenge mit den vom Untergrund bereinigten Ionenströmen und Depositionsdauern aufgetragen. Der gemittelte und vom Untergrund bereinigte Ionenstrom $I_{\text{clean}}$ wird über
\begin{align*}
  I_{\text{clean}}=\frac{I_{\text{vor}}+I_{\text{nach}}}{2}-I_{\text{Untergrund}}
\end{align*} 
berechnet. 
Es ergibt sich für die Belegung ein Wert von $\SI{200}{\pA\min}$, welcher sich mit der Annahme aus Kapitel \ref{sec:depo} in $0,15\,$ML übersetzt.
Mit dem realisierten Ionenstrom und der beobachteten Quellenstabilität konnte eine Belegung von einer Monolage im Zentrum des Depositionsflecks nicht erreicht werden.
Falls die Cluster optisch detektiert werden könnten, wären aber bei der Belegung von $0,15\,$Ml separierte Cluster im Zentrum zu erwarten.

\begin{table}
  \centering
  \caption{Abschätzung für die zweite Deposition. Augetragen sind die Ionenströme $I_{\text{vor}}$ vor und $I_{\text{nach}}$ nach einem Depositionsabschnitt, der Untergrund $I_{\text{Untergrund}}$, der gemittelte und vom Untergrund bereinigte Strom $I_{\text{clean}}$, die Zeit $t_{\text{Depo}}$ eines Depositionsabschnitt sowie die damit verbundene Belegung.}
  \label{tab:depo2}
  \begin{tabular}{llllll}
      \toprule
       $I_{\text{vor}}\,/\,\si{\pA}$	&	$I_{\text{nach}}$\,/\,\si{\pA}	& $I_{\text{Untergrund}}\,/\,\si{\pA}$		&	$I_{\text{clean}}\,/\,\si{\pA}$	&	$t_{\text{Depo}}\,/\,\si{\min}$	&	$\text{Belegung}\,/\,\si{\pA\min}$	\\
      \midrule
      1,13	&	1,45	&	0,3	&	0,99	&	60	&	59,4	\\
      1,19	&	0,62	&	0,3	&	0,605	&	108	&	65,3	\\
      1,15	&	0,87	&	0,25	&	0,76	&	50	&	38	\\
      0,9	&	0,75	&	0,2	&	0,625	&	61	&	38,1	\\
            &			&		&		&		&		\\
         		&		&		&		&	 	&	$\sum$ = 200,8	\\
      \bottomrule
  \end{tabular}
\end{table}


% Depo von großen Clustern siehe \ref{fig:cluster_big}
% \begin{figure}
%     \centering
%     \includegraphics[width=\textwidth]{./fig/cluster_big.pdf}
%     \caption{Caption cpation cpation.}
%     \label{fig:cluster_big}
% \end{figure}

\section{UV/vis Spektroskopie}
\label{sec:uvvis_ergebnisse}
Die UV/vis Spektroskopie wird im Rahmen dieser Arbeit genutzt um die Charakteristik des Depositionsflecks zu untersuchen.
Die Cluster konnten im Rahmen der optischen Messungen bisher nicht nachgewiesen werden.
Im Folgenden soll eine Abschätzung für das zu erwartende Signal gegeben werden.
Danach werden die optische Stabilität der Messung und die Homogenität des Probensystems untersucht.
Dies geschieht anhand einer Messreihe unter Verwendung einer $\SI{1}{\mm}$ Blende, siehe Abb. \ref{fig:uvvis_ablauf}.
% Da es nicht möglich war den Depositionsfleck zu finden wurde die Probe, unter Verwendung einer $\SI{1}{\mm}$ Blende abgerastert, siehe Abb. \ref{fig:uvvis_ablauf}.
% \begin{figure}
%     \centering
%     \includegraphics[width=\textwidth]{./fig/uvvis_raster.pdf}
%     \caption{l. raster, oben r. Glasträger mit ITO / Probe ITO in sample holder bild raus
%      zeilenweise gerastert sagen dass x8 y15 mehrmals aufgenommen wurde, refrenzstelle für optische stabilität    Caption Caption Caption Caption Caption Caption Caption .}
%     \label{fig:uvvis_raster}
% \end{figure}

\begin{figure}
    \begin{subfigure}[b]{0.65\textwidth}
      \includegraphics[width=\textwidth]{./fig/uvvis_raster.pdf}
      \caption{}
      \label{fig:uvvis_raster}
    \end{subfigure}\hfill
    \begin{subfigure}[b]{0.325\textwidth}
      \includegraphics[width=\textwidth]{./fig/uvvis_probe.png}
      \caption{}
      \label{fig:uvvis_probe}
    \end{subfigure}
    \caption{(a) Die Probe wurde abgerastert und nach jeder Zeile die Referenzstelle erneut angefahren. Das Signal des Referenzpunktes ist in Abb. \ref{fig:uvvis_ref} dargestellt. Der Farbton, den eine Zeile aufweist, z.B erste Zeile dunkelrot, entspricht dem dunkelroten Spektrum in Abb. \ref{fig:uvvis_ref}. Das Signal des Referenzpunktes, nachdem die zweite Zeile (helles rot) durchlaufen wurde, entspricht dem hellroten Signal in Abb. \ref{fig:uvvis_ref} und so weiter. Die gesamte Messdauer betrug ca. $\SI{90}{\min}$. (b) ITO Glasträger der zweiten Deposition befestigt auf einem weiteren Glasträger, welcher in dem UV/vis Aufbau eingespannt wird. In Grün eingezeichnet ist der Bereich, in dem die Probe abgerastert wurde.} 
    \label{fig:uvvis_ablauf}
\end{figure}
% Die Integrationszeit eines einzelnen Scans war $\SI{8}{\ms}$ und für eine Spektrum wurde über 500 Scans gemittelt.
\subsection{Scout Rechnung}
\label{sec:scoutsim}
% abschätzung für messung nötig, zb kreibig vollmer 
Im Falle der UV/vis Spektroskopie ist die Wellenlänge des Lichts viel größer als die Abmessungen der hier untersuchten Cluster und das Probensystem kann als effektives Medium betrachtet werden \cite[S. 149 ff.]{Thei.1993}.
Die von Maxwell Garnett aufgestellte Formel für das effektive Medium beschreibt kugelförmige Partikel mit großem Abstand zueinander und ist somit einge gute Näherung für eine niedrige Clusterbelegung. 
Für eine Abschätzung des zu erwartenden Absorptionsignals wurde ein $\SI{3}{\nm}$ dicker Film als effektives Medium in Vakuum mit Maxwell Garnett für verschiedene Füllfaktoren berechnet.
Ein Füllfaktor von $0,075$ entspricht dabei $0,15$ Clustermonolagen.
Die Berechnungen wurden von Alexander Kononov zur Verfügung gestellt und sind in Abb. \ref{fig:scout} zu sehen.\\
Die erwartete Absorption für eine Belegung von $0,15$ Clustermonolagen, wie sie für das zweite Depositionsexperiment abgeschätzt wurde, siehe Tabelle \ref{tab:depo2}, beläuft sich auf $0,3 - 0,6\,\%$.
Für kleine Wellenlängen ist dabei der größte Effekt zu erwarten und bei $\SI{400}{\nm}$ ergibt sich aus den Berechnungen noch eine Absorption von $0,4\,\%$. Die Empfindlichkeit der optischen Messungen sollte daher im Subprozent-Bereich sein.
\begin{figure}[h]
  \centering
  \includegraphics[width=0.9\textwidth]{./fig/scout.pdf}
  \caption{Abgebildet ist das berechnete Absorptionsignal eines $\SI{3}{\nm}$ dicken Eisen Films in Vakuum, als effektives Medium mit Maxwell Garnett für verschiedene Füllfaktoren. Ein Füllfaktor von $0,075$ entspricht dabei $0,15$ Clustermonolagen. Rechnungen bereitgestellt von Alexander Kononov.}
  \label{fig:scout}
\end{figure}

\subsection{Optische Stabilität}
\label{sec:uvvis_stabilitaet}
Wie in Kapitel \ref{sec:scoutsim} beschrieben, wird eine Absorption von weniger als einem Prozent erwartet.
Ein systematischer Störeffekt während der Messung ist die Veränderung der Intensität der Lampe.
% Dies führt dazu, dass auf der gleichen Probenstelle eine veränderte Absorption gemessen wird.
Daher wurde bei dem Rastern der Probe nach jeder Zeile die Referenzstelle wiederholt gemessen.
In Abbildung \ref{fig:uvvis_ref} ist diese Veränderung innerhalb einer Messreihe zu sehen.
Das erste und letzte Spektrum liegen dabei etwa $\SI{90}{\min}$ auseinander und die Lampe wurde ca. drei Stunden vor Beginn der Messreihe eingeschaltet.
\begin{figure}
    \centering
    \includegraphics[width=0.8\textwidth]{./fig/uvvis_ref.png}
    \caption{Abgebildet sind die Spektren der Referenzstelle (siehe \ref{fig:uvvis_raster}) im Laufe der Messung. In den etwa $\SI{90}{\min}$ stieg die Intensität der Lampe, daher wird auf derselben Probenposition mit der Zeit eine veränderte Absorption gemessen.
    Die Absorption startete bei null und weichte nach der letzten Messung um bis zu $\SI{0.3}{\%}$ ab. Die Spektren entsprechen den Messungen des Referenzspots, nach der jeweiligen Zeilen im gleichen Farbton aus Abb. \ref{fig:uvvis_raster}.}
    \label{fig:uvvis_ref}
\end{figure}
Die Intensität der Lampe ist im Laufe der Messung größer geworden und übersteigt die der ersten Referenzmessung. 
Dies führt zu einer erhöhten Transmission, bzw. einer negativen Absorption in der Größenordnung von $\SI{0.3}{\%}$
Mit diesem Verhalten würde auch bei einem ideal homogenen Substrat eine Struktur zu erkennen sein.
Daher wird für die weitere Auswertung immer die aktuellste Referenz verwendet und so der Fehler in der Absorption verringert.
% Dadurch verringert sich der Fehler in der Absorption 
\subsection{Probeninhomogenität}
In Abb. \ref{fig:uvvis_map} ist die integrierte Absorption der Spektren im Bereich von $\SI{350}{\nm}$ bis $\SI{500}{\nm}$ an den jeweiligen Probenstellen aufgetragen, vgl Abb. \ref{fig:uvvis_ablauf}.
Dadurch werden die lokalen Inhomogenintäten des Substrats sichtbar.
Die starke Absorption bei $x=\SI{7}{\mm}$ und $y=\SI{8}{\mm}$ lässt sich durch einen Beschädigung des Substrats erklären, welche mit einem Lichtmikroskop zu erkennen war.
Im mittleren Probenbereich sind die Abweichungen der Absorption in einer Größenordung von etwa $\SI{1.5}{\%}$ und damit größer als das aus den Berechnungen erwartete Signal der Cluster, vgl \ref{fig:scout}.
\begin{figure}
    \centering
    \includegraphics[width=\textwidth]{./fig/uvvis_map.png}
    \caption{Abgebildet ist die integrierte Absorption der Spektren im Bereich von $\SI{350}{\nm}$ bis $\SI{500}{\nm}$. An den weißen Stellen gibt es keine Messungen, vgl Abb. \ref{fig:uvvis_ablauf}.}
    \label{fig:uvvis_map}
\end{figure}
% \subsection{ITO Transmission}
% Verschiedene Anzahl Schichten und ggf verschieben

% \section{AFM}
% Ito lässt sich abbilden, Bild dazu, Aussage über Rauhigkeit treffen, genügend glatt??






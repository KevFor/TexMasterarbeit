\chapter{Ergebnisse}
In diesem Kapitel sind die Massenspektren für kleine und große Cluster aufgeführt.
Anhand des stabilen Signals von Fe$_1$ wurde der Fokus des Clusterstrahls untersucht.
% Einige Schwierigkeiten bei der Inbetriebnahme bzw bei längerer Inbetriebnahme
Mit der erfolgreichen Inbetriebnahme der Schleusenkammer ist die Deposition auf ITO möglich und Depositionsmenge wird abgeschätzt.
Nach der Deposition folgten \textit{ex situ} UV/vis Spektroskopie Messungen.

% auch AFM Messungen aber nur Substrat, hier Cluster nicht abgebildet
\section{Massenspektren bzw Strom}
Nur FC oder auch Chtr? \ref{fig:cluster_chtr}
Massenspektren kleine und große Cluster
Strom kann auch 2 pA sein, siehe Quelltext, name des Spektrums %2018-11-21_04_mass1000-250000_wait500_step1000_tw15_FC2_Quelle0cm_Iris9,08_13Watt
Strom 4 pA für Fe1 siehe FC messung in Abh von x
hier Spektren bzw Strom in Chtr bzw FC2 zeigen. Optiken funktionieren, anschauen von Fokus von Fe1 im FC2.
Für Strahlfokus die entsprechenden Abbildungen zeigen.
\ref{fig:cluster_chtr}

\begin{figure}
    \begin{subfigure}[t]{0.475\textwidth}
      \includegraphics[width=\textwidth]{./fig/cluster_small.pdf}
      \caption{Caption....}
      \label{fig:cluster_small}
    \end{subfigure}\hfill
    \begin{subfigure}[t]{0.475\textwidth}
      \includegraphics[width=\textwidth]{./fig/cluster_medium.pdf}
      \caption{Caption..}
      \label{fig:cluster_medium}
    \end{subfigure}
    \caption{\ref{fig:cluster_small} und \ref{fig:cluster_medium} Clsuterstrom in chtr für kleiner Cluster bis $Fe_{30}$.} 
    \label{fig:cluster_chtr}
\end{figure}



\section{Strahlfokus}
Mit dem stabilen Signals von Fe$_1$ wurde der Fokus des Clusterstrahls untersucht.
Für große Cluster war dies nicht möglich weil da Signal zu Untergrund Verhältnis zu schlecht war.

messdaten = faltung von realem strahlprofil mit fc öffnung
vergleich von messdaten mit faltung 

So wie in Vortrag zeigen, dass hier nur FC1 Strom
Hier Fokus \ref{fig:faltung} statt unter Kapitel Simulation
\begin{figure}
    \centering
    \includegraphics[scale=0.9]{./fig/faltung.pdf}
    \caption{Abbildung überarbeiten und genau beschreiben oben mitte unten jewiels gneau.}
    \label{fig:faltung}
\end{figure}

\section{Probleme? schwankungen, strom läuft weg - oxidieren}
Stabilität des Stroms
Schwanken des Stroms (Kryo), vor und nach Depo bzw zwischendurch kontrollieren - 
Strom sinkt bzw schiebt zu anderen Massen (thermisch bzw Optiken-Iris, Skimmer beschichtet)

Mehr Strom/Depodauer nötig (Diskussion), gestaltet sich aber schwierig das umzusetzen.. 
oxidieren justage strom haut ab blabla


\section{Deposition}
Sagen das 2 mal Depo, erstes mal aber hinterher kein Strom, nicht klar wann weg
zweite depo nach 1 std strom kontrolliert und an 2 tagen deponiert weil Strom ganz weg gelaufen ist.
Abschätzung wie viel belegt - Strom (vorher nachher) minus Untergrund und Depospot größe => x CML
hier etwas genauer Deposioton erlären, pAmin erklären etc, 2 Tage vorher nacher Strom 

Depo von großen Clustern siehe \ref{fig:cluster_big}
\begin{figure}
    \centering
    \includegraphics[scale=0.9]{./fig/cluster_big.pdf}
    \caption{Caption cpation cpation.}
    \label{fig:cluster_big}
\end{figure}

\newpage
\section{UV/vis Spektroskopie}
Die UV/vis Spektroskopie wird im Rahmen dieser Arbeit genutzt um den Depositionsfleck zu finden.
Dafür wurde der.....
Ablauf \ref{fig:uvvis_ablauf} vlt in anderem/früheren Kapitel

% \begin{figure}
%     \centering
%     \includegraphics[scale=0.6]{./fig/uvvis_raster.png}
%     \caption{l. raster, oben r. Glasträger mit ITO / Probe ITO in sample holder bild raus
%      zeilenweise gerastert sagen dass x8 y15 mehrmals aufgenommen wurde, refrenzstelle für optische stabilität    Caption Caption Caption Caption Caption Caption Caption .}
%     \label{fig:uvvis_raster}
% \end{figure}

\begin{figure}
    \begin{subfigure}[t]{0.65\textwidth}
      \includegraphics[width=\textwidth]{./fig/uvvis_raster.png}
      \caption{l. raster, oben r. Glasträger mit ITO / Probe ITO in sample holder bild raus
      zeilenweise gerastert sagen dass x8 y15 mehrmals aufgenommen wurde, refrenzstelle für optische stabilität    Caption Caption Caption Caption Caption Caption Caption .}
      \label{fig:uvvis_raster}
    \end{subfigure}\hfill
    \begin{subfigure}[t]{0.325\textwidth}
      \includegraphics[width=\textwidth]{./fig/uvvis_probe.png}
      \caption{ITO auf Glasträger ....}
      \label{fig:uvvis_probe}
    \end{subfigure}
    \caption{\ref{fig:uvvis_raster} Ablauf und \ref{fig:uvvis_probe} Probe.......} 
    \label{fig:uvvis_ablauf}
\end{figure}

\subsection{Scout Rechnung}
\label{sec:scoutsim}
Hier Scout von Alex \ref{fig:scout}
Erwartete Absorption für Eisen Film als effektives Medium in Vakuum mit maxwell garnett \cite{Thei.1993} berechnet von Alex Kononov mit Scout

\begin{figure}
    \centering
    \includegraphics[scale=0.8]{./fig/scout.png}
    \caption{Caption Caption Caption Caption Caption Caption Caption .}
    \label{fig:scout}
\end{figure}

\subsection{Optische Stabilität}
Wie in Kapitel \ref{sec:scoutsim} beschrieben, sollten die Eisen Cluster eine Absorption von weniger als ein Prozent bewirken.
Ein Störeffekt der bei der UV/vis Spektroskopie auftreten kann ist die Veränderung der Intensität der Lampe. 
% Dies führt dazu, dass auf der gleichen Probenstelle eine veränderte Absorption gemessen wird.
Daher wurde bei dem Rastern der Probe nach jeder Zeile die Referenzstelle wiederholt gemessen.
In Abbildung \ref{fig:uvvis_ref} ist diese Veränderung innerhalb einer Messreihe zu sehen.
Das erste und letzte Spektrum liegen dabei etwa eine Stunde außeinander und die Lampe wurde ca. drei Stunden vor Beginn der Messreihe eingeschaltet.
\begin{figure}
    \centering
    \includegraphics[scale=0.8]{./fig/uvvis_ref.png}
    \caption{Abbildungs schöner machen.. achsenbeschriftung label etc., y-Skala zu Prozent machen Caption überarbeiten
    Absorptionsmessungen einer Referenzstelle im Laufe einer Messreihe.
    Die Spektren entsprechen den jeweiligen Zeilen im gleichen Farbton aus Abb. \ref{fig:uvvis_raster}.}
    \label{fig:uvvis_ref}
\end{figure}
Die Intensität der Lampe ist im Laufe der Messung größer geworden und übersteigt die der Referenzmessung $I_Ref$. 
Dies führt zu einer erhöhten Transmission, bzw. einer negativen Absorption.

beschreiben steigen von I null
größenordnung und dmait Messgenauigkeit hinsichtlich Lampe







Referenz im Laufe der Messung zeigen.
Keine Cluster im Rahmen der Messgenauigkeit zu sehen.

\subsection{Probeninhomogenität}
Absorption bzgl referenzstelle xy 
Absorption im Bereich 550 bis 650nm an den jeweiligen Probenstellen aufgetragen


Hier Heatmap und Kratzer zeigen
Messgenauigkeit hinsichtlich Probeninhomogenität
Zum einen weglaufen Referenz, Probeninhomogenitäten und starke Abweichungen wie Kratzer (mit Lichtmikroskop auflösbar - vergrößerung bzw größe Kratzer sagen - erkennbar wärhend UV/vis(bloßes Auge) bzw Lichtmikroskop, nicht ohne Hilfsmittel)
Inhomegintät eine Prozent so wie Signal
Kratzer deutlich stärker, mehrere Prozent / deutlich mehr als 1 Prozent in Absorption

\begin{figure}
    \centering
    \includegraphics[scale=0.7]{./fig/uvvis_map.png}
    \caption{Caption Caption Caption Caption Caption Caption Caption Caption Caption Caption Caption Caption Caption Caption Caption Caption Caption Caption .}
    \label{fig:uvvis_map}
\end{figure}

\section{AFM}
Ito lässt sich abbilden, Bild dazu, Aussage über Rauhigkeit treffen, genügend glatt?? \


\subsection{ITO Transmission}

Verschiedene Anzahl Schichten und ggf verschieben

\chapter{Experimenteller Aufbau}
Für die Untersuchung und Erzeugung von Clustern gibt es einen Aufbau der sich in zwei Komponenten unterteilen lässt, vgl. Abb \ref{fig:aufbau}.
Im Rahmen dieser Arbeit waren das Präparations-/Analysesystem und die Clusterstrahlanlage (kurz CSA) voneinander getrennt, mehr Informationen zum bisherigen Aufbau finden sich beispielsweise in \cite{wolter}.
Hinter der CSA wird anstelle des Analysesystems die Schleusen Kammer in Betrieb genommen.
% Mit dieser soll die Erzeugung und schnelle Entnahme von Proben, für optische Messungen, realisiert werden.
% CSA ruhig kurz halten und auf Dominik und Yin verweisen, ggf andere die irgendwas speziell gemacht haben
\begin{figure}
    \centering
    \includegraphics[width=\textwidth]{./fig/aufbau.png}
    \caption{Experimenteller Aufbau mit der bisher genutzten Präparationskammer und den drei Abschnitten der Clusterstrahlanlage, \cite[S. 20]{wolter} adaptiert. 
    In grün ist der Verlauf des Clusterstrahls eingezeichnet.}
    \label{fig:csaaufbau}
\end{figure}

\section{Clusterstrahlanlage}
Die Clusterstrahlanlage kann in drei Abschnitte unterteilt werden, die Clusterquelle, die Kryo Kammer und den Massenselektor, vgl. Abb \ref{fig:aufbau}.
In diesen werden die Cluster erzeugt, der Clusterstrahl fokussiert und die gewünschte Clustergröße eingestellt. Die genauere Funktionsweise der einzelnen Elemente wird im folgenden beschrieben.
\subsection{Clusterquelle}
Prinzip der Erzeugung ist das.
\subsection{Kryo Kammer}
auf Diplomarbeit krause verweisen, gut erklärt
Fokussierung und Restgas weg.
Duffe und dowo
\subsection{Massenselektor}
Gewünschte Clustergröße einstellen 
Massenselektor \cite{Issendorff.1999}
Bild von Chtr zeigen, ablenkung des Strahls
\begin{figure}
    \centering
    \includegraphics[width=\textwidth]{./fig/massenselektor_2.png}
    \caption{Hier Bild MS, insbersondere Chtr und elektrode zum umschalten.}
    \label{fig:mschtr}
\end{figure}
\section{Schleusen Kammer}
\label{sec:schleuse}

kurz beschreiben und auf nächste Kapitel hinweisen - Simulaton bzw Konstruktion
Foto beschriften, ggfg besseres Foto?
% Mit dieser soll die Erzeugung und schnelle Entnahme von Proben, für optische Messungen, realisiert werden.
% CSA ruhig kurz halten und auf Dominik und Yin verweisen, ggf andere die irgendwas speziell gemacht haben
\begin{figure}
    \centering
    \includegraphics[width=\textwidth]{./fig/schleuse.jpg}
    \caption{caption.}
    \label{fig:schleuse}
\end{figure}


\section{UV/vis und AFM Aufbau erklären}
UV/vis und ggf AFM hier erklären? wohin, afm überhaupt?
Foto von UV/vis Aufbau und ggf hier Messablauf erklären

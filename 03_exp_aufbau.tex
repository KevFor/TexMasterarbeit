\chapter{Experimenteller Aufbau}
Für die Untersuchung und Erzeugung von Clustern gibt es einen Aufbau der sich in zwei Komponenten unterteilen lässt, vgl. Abb \ref{fig:csaaufbau}.
Im Rahmen dieser Arbeit waren das Präparations-/Analysesystem und die Clusterstrahlanlage (kurz CSA) voneinander getrennt, mehr Informationen zum bisherigen Aufbau finden sich beispielsweise in \cite{wolter}.
Hinter der CSA wird anstelle des Analysesystems die Schleusen Kammer in Betrieb genommen.
% Mit dieser soll die Erzeugung und schnelle Entnahme von Proben, für optische Messungen, realisiert werden.
% CSA ruhig kurz halten und auf Dominik und Yin verweisen, ggf andere die irgendwas speziell gemacht haben
\begin{figure}
    \centering
    \includegraphics[width=\textwidth]{./fig/aufbau.png}
    \caption{Experimenteller Aufbau mit der bisher genutzten Präparationskammer und den drei Abschnitten der Clusterstrahlanlage, \cite[S. 20]{wolter} adaptiert. 
    In grün ist der Verlauf des Clusterstrahls eingezeichnet.}
    \label{fig:csaaufbau}
\end{figure}

\section{Clusterstrahlanlage}
Die Clusterstrahlanlage kann in drei Abschnitte unterteilt werden, die Clusterquelle, die Kryo Kammer und den Massenselektor, vgl. Abb \ref{fig:csaaufbau}.
In diesen werden die Cluster erzeugt, der Clusterstrahl fokussiert und die gewünschte Clustergröße eingestellt.
Die genauere Funktionsweise der einzelnen Elemente wird im folgenden beschrieben, mehr Information finden sich z.B. in den Arbeiten von \cite{duffe, wolter, schröder}.
\subsection{Clusterquelle}
Für die Erzeugung der Cluster wird eine Magnetron-Sputter-Gas-Aggregations-Quelle benutzt, bei der aus einem Target Atome gesputtert werden, welche dann in einer Edelgasatmosphäre zu Clustern kondensieren, vgl Abb. \ref{fig:quelle}.
Das Target ist auf einem Magnetron Kopf montiert, durch einlassen eines Argon-Helium Gemisches und Anlegen einer Spannung von etwa $\SI{1}{\kV}$ wird ein Plasma gezündet.
Dadurch werden die Atome gesputtert und können dann in dem Aggregationsvolumen zu Clustern kondensieren, bis zu $80\,\%$ liegen als Ionen vor \cite{Haberland.1991}.
Dabei wird ein breites Spektrum an Clustern erzeugt und die Verteilung kann durch verschiedene Parameter varieert werden.
Diese sind die Menge und Zusammensetzung des Gas-Gemisches, die Leistung mit der gesputtert wird, das Aggregationsvolumen sowie die Potentiale der Ionenoptiken in der Quelle.
\begin{figure}
    \centering
    \includegraphics[width=\textwidth]{./fig/quelle.png}
    \caption{Schematischer Aufbau der Clusterquelle, \cite{woltermaster}.}
    \label{fig:quelle}
\end{figure}
\subsection{Kryo Kammer}
Nach der Quelle gelangen die Cluster in die Kryokammer, hier wird der diffuse Strahlanteil mit einem Skimmer abgeschnitten und mit Ionenoptiken der Strahl fokussiert und auf $\SI{500}{\eV}$ beschleunigt, siehe Abb. \ref{fig:kryokammer}.
Der darauf folgende X/Y-Deflektor ermöglicht eine Verkippung des Strahls senkrecht zur Strahlrichtung.
Für die Bereinigung des Ionenstrahls vom Restgas wird mit einer Kryopumpe eine Temperatur von etwa $\SI{10}{\kelvin}$ \cite[S. 27]{woltermaster} an der ersten Kryostufe erzeugt, dadurch kann das austretende Argon ausgefroren werden.
Weitere Informationen zu den einzelnen Komponenten finden sich in \cite{krause}.
\begin{figure}
    \centering
    \includegraphics[width=\textwidth]{./fig/kryokammer.png}
    \caption{Schematischer Aufbau der Kryokammer, \cite{woltermaster}.}
    \label{fig:kryokammer}
\end{figure}

\subsection{Massenselektor}
In dem Massenselektor, entwickelt von \cite{Issendorff.1999}, wird der Ionenstrahl durch Anlegen von zeitlich begrenzten Hochspannungspulsen senkrecht zur Strahlrichtung abgelenkt und dadurch eine Aufspaltung des Strahls bewirkt.
Durch den Puls erhalten alle Cluster die gleiche Energie, kleinere Cluster haben weger ihrer geringeren Masse eine höhere Geschwindigkeit.

Gewünschte Clustergröße einstellen 

\begin{figure}
    \centering
    \includegraphics[width=\textwidth]{./fig/massenselektor_3.png}
    \caption{Hier Bild MS, insbersondere Chtr und elektrode zum umschalten.}
    \label{fig:mschtr}
\end{figure}


\section{Schleusen Kammer}
\label{sec:schleuse}

kurz beschreiben und auf nächste Kapitel hinweisen - Simulaton bzw Konstruktion
Foto beschriften, ggfg besseres Foto?
% Mit dieser soll die Erzeugung und schnelle Entnahme von Proben, für optische Messungen, realisiert werden.
% CSA ruhig kurz halten und auf Dominik und Yin verweisen, ggf andere die irgendwas speziell gemacht haben
\begin{figure}
    \centering
    \includegraphics[width=\textwidth]{./fig/schleuse.jpg}
    \caption{caption.}
    \label{fig:schleuse}
\end{figure}


\section{UV/vis und AFM Aufbau erklären}
UV/vis und ggf AFM hier erklären? wohin, afm überhaupt?
Foto von UV/vis Aufbau und ggf hier Messablauf erklären

\chapter{Diskussion und Ausblick}
% Zu Strahlfokus, bisher nur Fe$_1$ Fokus, sofern bei anderen Ionen besseres Signal-rausch-Verhältnis, dann auch hier Fokus anschauen.
\section{Inbetriebnahme}
Für die Schleusenkammer wurden neue Ionenoptiken konstruiert, simuliert und erfolgreich in Betrieb genommen.
Die Simulation zeigten auf, dass ohne Depositionsoptiken keine stabile Strahlführung möglich ist.
In Abschnitt \ref{sec:fokus1} ergab sich für den konvergenten bzw. divergenten Strahl auf dem Faraday cup ein Fokus mit weniger als $\SI{0,3}{\mm}$ bzw. $\SI{2}{\mm}$ Durchmesser.
Durch den Einbau von Depotube sowie Probenhalter+Blende war es möglich den Ionenstrahl im neuen Aufbau zu detektieren.
Für große Eisencluster, hier $Fe_{1790}$, ergab sich ein maximaler Clusterstrom von $\SI{2}{\pA}$ welche jedoch nicht über mehrere Stunden stabil blieb.\\
Mit dem Signal von $Fe_1$ wurde der Fokus vermessen und für das Strahlprofil, unter Annahme einer Gauß-Verteilung, eine Halbwertsbreite von $\SI{1,5}{\milli\meter}$ bestimmt.
Bei der Verwendung des Oberflächenanalyse System und den entsprechenden Depositionsoptiken hatte der Depositionspot einen Durchmesser von etwa $\SI{1}{\mm}$ \cite[S. 40]{gronhagen}.
Der neue Aufbau bestehend aus Depotube und Blende ermöglicht somit eine angemessene Fokussierung der Clusterstrahls.
% ....vergleich zu simulation...
\section{Große Eisencluster auf ITO}
Mit großen Eisenclustern wurden erste Depositionsexperimente auf ITO durchgeführt und eine Probe mit einer Belegung von $0,15$ Clustermonolagen erzeugt.
Berechnungen für diese Depositionsmenge ergaben eine erwartete Absorption von $0,3 - 0,6\,\%$, siehe Abschnitt \ref{sec:scoutsim}.\\
Die Messungen der Probe zeigten lokale Inhomogenitäten des Substrats auf und die Abweichungen des Absorption sind etwa $\SI{1.5}{\%}$.
Die Charakterisierung des Depositionsflecks war daher bisher nicht möglich

.....
optische Messungen verbessern

Leerrasterung und dann gleiche Stelle nach Depo anschauen - Problem Tage/Wochen abstand zwischen Leermessung und nach Depo

% Untergrund auch bei sehr niedrigem druck von $\SI{2e-10}{}\,\text{mbar}$ \cite{Senz.2009}

% Für die Untersuchung von großen Eisen Clustern wird eine Belegung von $0,1-0,2$ Monolagen angestrebt, weil dann die relalive Zahl koaleszierter Cluster weniger als 40 \% betragen sollte\cite[S. 88]{bsieben}.
% Für das Profil des Ionenstrahls wird eine Gauss Verteilung erwartet, daher sollten auch bei höheren Belegungen die Randbereiche des Depositionsflecks genügend separierte Cluster aufweisen.
% Für das Auffinden des Depospots bietet sich eine höhere Belegung von einer Monolage an, weil die erwartete Absorption dann auch größer ist.

%\chapter{Ausblick}

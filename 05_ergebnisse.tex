\chapter{Große Eisencluster auf ITO}
% Einige Schwierigkeiten bei der Inbetriebnahme bzw bei längerer Inbetriebnahme
Mit der erfolgreichen Inbetriebnahme der Schleusenkammer ist die Deposition auf ITO möglich und die Depositionsmenge wird abgeschätzt.
Nach der Deposition folgten \textit{ex situ} UV/vis Spektroskopie Messungen. 

% auch AFM Messungen aber nur Substrat, hier Cluster nicht abgebildet

\section{Deposition}



Sagen das 2 mal Depo, erstes mal aber hinterher kein Strom, nicht klar wann weg
zweite depo nach 1 std strom kontrolliert und an 2 tagen deponiert weil Strom ganz weg gelaufen ist.
Abschätzung wie viel belegt - Strom (vorher nachher) minus Untergrund und Depospot größe => x CML
hier etwas genauer Deposioton erlären, pAmin erklären etc, 2 Tage vorher nacher Strom 

Depo von großen Clustern siehe \ref{fig:cluster_big}
\begin{figure}
    \centering
    \includegraphics[scale=0.8]{./fig/cluster_big.pdf}
    \caption{Caption cpation cpation.}
    \label{fig:cluster_big}
\end{figure}

\section{UV/vis Spektroskopie}
Die UV/vis Spektroskopie wird im Rahmen dieser Arbeit genutzt um die Position und die Größe des Depositionsflecks zu untersuchen.
Da es nicht möglich war den Depositionsfleck auf Anhieb zu finden wurde die Probe, unter Verwendung einer $\SI{1}{\mm}$ Blende abgerastert, siehe Abb. \ref{fig:uvvis_ablauf}.
% \begin{figure}
%     \centering
%     \includegraphics[scale=0.6]{./fig/uvvis_raster.pdf}
%     \caption{l. raster, oben r. Glasträger mit ITO / Probe ITO in sample holder bild raus
%      zeilenweise gerastert sagen dass x8 y15 mehrmals aufgenommen wurde, refrenzstelle für optische stabilität    Caption Caption Caption Caption Caption Caption Caption .}
%     \label{fig:uvvis_raster}
% \end{figure}

\begin{figure}
    \begin{subfigure}[b]{0.65\textwidth}
      \includegraphics[width=\textwidth]{./fig/uvvis_raster.pdf}
      \caption{Die Probe wurde abgerastert und nach jeder Zeile die Referenzstelle erneut angefahren. 
      Die gesamte Messdauer betrug ca. $\SI{90}{\min}$ und die Farbtöne zeigen auf, zu welchem Zeitpunkt die Referenzspektren aufgenommen wurden, siehe Abb. \ref{sec:scoutsim}.}
      \label{fig:uvvis_raster}
    \end{subfigure}\hfill
    \begin{subfigure}[b]{0.325\textwidth}
      \includegraphics[width=\textwidth]{./fig/uvvis_probe.png}
      \caption{ITO Glasträger befestigt auf einem weiteren Glasträger welcher in dem UV/vis Aufbau eingespannt wird. In grün eingezeichnet ist der Bereich in dem die Probe abgerastert wurde.}
      \label{fig:uvvis_probe}
    \end{subfigure}
    \caption{Messablauf \ref{fig:uvvis_raster} der Spektroskopie und die Probe \ref{fig:uvvis_probe}.} 
    \label{fig:uvvis_ablauf}
\end{figure}
% Die Integrationszeit eines einzelnen Scans war $\SI{8}{\ms}$ und für eine Spektrum wurde über 500 Scans gemittelt.

\subsection{Optische Stabilität}
Wie in Kapitel \ref{sec:scoutsim} beschrieben, wird eine Absorption von weniger als einem Prozent erwartet.
Ein systematischer Störeffekt bei der Absorption ist die Veränderung der Intensität der Lampe.
% Dies führt dazu, dass auf der gleichen Probenstelle eine veränderte Absorption gemessen wird.
Daher wurde bei dem Rastern der Probe nach jeder Zeile die Referenzstelle wiederholt gemessen.
In Abbildung \ref{fig:uvvis_ref} ist diese Veränderung innerhalb einer Messreihe zu sehen.
Das erste und letzte Spektrum liegen dabei etwa $\SI{90}{\min}$ außeinander und die Lampe wurde ca. drei Stunden vor Beginn der Messreihe eingeschaltet.
\begin{figure}
    \centering
    \includegraphics[scale=0.7]{./fig/uvvis_ref.png}
    \caption{Abbildungs schöner machen.. achsenbeschriftung label etc., y-Skala zu Prozent machen Caption überarbeiten
    Absorptionsmessungen einer Referenzstelle im Laufe einer Messreihe.
    Die Spektren entsprechen den jeweiligen Zeilen im gleichen Farbton aus Abb. \ref{fig:uvvis_raster}.}
    \label{fig:uvvis_ref}
\end{figure}
Die Intensität der Lampe ist im Laufe der Messung größer geworden und übersteigt die der ersten Referenzmessung. 
Dies führt zu einer erhöhten Transmission, bzw. einer negativen Absorption in der Größenordung von $\SI{0.3}{\%}$
Mit diesem Verhalten würde auch bei einem ideal homogenen Substat eine Struktur zu erkennen sein.
Daher wird für die weitere Auswertung immer die aktuellste Referenz verwendet und so der Fehler in der Absorption verringert.
% Dadurch verringert sich der Fehler in der Absorption 
\subsection{Probeninhomogenität}
In Abb. \ref{fig:uvvis_map} ist die integrierte Absorption der Spektren im Bereich von $\SI{350}{\nm}$ bis $\SI{500}{\nm}$ an den jeweiligen Probenstellen aufgetragen, vgl Abb. \ref{fig:uvvis_ablauf}.
Dadurch werden die lokalen Inhomegintäten des Substrats sichtbar.
Die starke Absorption bei $x=\SI{7}{\mm}$ und $y=\SI{8}{\mm}$ lässt sich durch einen Beschädigung des Substrats erklären welche mit einem Lichtmikroskop zu erkennen war.
Im mittleren Probenbereich sind die Abweichungen der Absorsption in einer Größenordung von etwa $\SI{1.5}{\%}$ und damit größer als das aus den Berechnungen erwartete Signal, vgl \ref{fig:scout}.
\begin{figure}
    \centering
    \includegraphics[scale=0.7]{./fig/uvvis_map.png}
    \caption{Abgebildet ist die integrierte Absorption der Spektren im Bereich von $\SI{350}{\nm}$ bis $\SI{500}{\nm}$. An den weißen Stellen gibt es keine Messungen, vgl Abb. \ref{fig:uvvis_ablauf}.}
    \label{fig:uvvis_map}
\end{figure}
% \subsection{ITO Transmission}
% Verschiedene Anzahl Schichten und ggf verschieben

% \section{AFM}
% Ito lässt sich abbilden, Bild dazu, Aussage über Rauhigkeit treffen, genügend glatt??






\chapter{Ergebnisse}
\section{Massenspektren bzw Strom}
hier Spektren bzw Strom in Chtr bzw FC2 zeigen. Optiken funktionieren, anschauen von Fokus von Fe1 im FC2.
\section{Strahlfokus}
Hier Fokus statt unter Kapitel Simulation
\section{Probleme? schwankungen, strom läuft weg - oxidieren}
Stabilität des Stroms
Schwanken des Stroms, vor und nach Depo bzw zwischendurch kontrollieren - Strom sinkt bzw schiebt zu anderen Massen
\section{UV/vis Spektroskopie}
Die UV/vis Spektroskopie wird im Rahmen dieser Arbeit genutzt um den Depositionsfleck zu finden. 


\subsection{Optische Stabilität}
Wie in Kapitel \ref{sec:scoutsim} beschrieben, sollten die Eisen Cluster eine Absorption von weniger als ein Prozent bewirken.
Ein Störeffekt der bei der UV/vis Spektroskopie auftreten kann ist die Veränderung der Intensität der Lampe. 
Dies führt dazu, dass auf der gleichen Probenstelle eine veränderte Absorption gemessen wird.
Daher wurde bei dem Rastern der Probe in regelmäßigen Abständen eine Referenzstelle wiederholt gemessen.
In Abbildung \ref{fig:uvvis_ref} ist diese Veränderung innerhalb einer Messreihe zu sehen.
Das erste und letzte Spektrum liegen dabei etwa eine Stunde außeinander und die Lampe wurde ca. drei Stunden vor Beginn der Messreihe eingeschaltet. \\
\begin{figure}
    \centering
    \includegraphics[scale=0.8]{./fig/uvvis_ref.png}
    \caption{Abbildungs schöner machen.. achsenbeschriftung label etc., y-Skala zu Prozent machen Caption Caption Caption Caption Caption Caption Caption Caption Caption Caption Caption Caption Caption Caption Caption Caption Caption Caption .}
    \label{fig:uvvis_ref}
\end{figure}
Die Intensität der Lampe nimmt zu und übersteigt die der Referenzmessung $I_0$ und bewirkt somit eine negative Absorption. 

beschreiben steigen von I null
größenordnung und dmait messgenauigkeit







Referenz im Laufe der Messung zeigen.
Keine Cluster im Rahmen der Messgenauigkeit zu sehen.

\subsection{Probeninhomegenität}

Ablauf \ref{fig:uvvis_raster} nur vlt bzw in anderem/früheren Kapitel
\blindtext
\begin{figure}
    \centering
    \includegraphics[scale=0.8]{./fig/uvvis_raster.png}
    \caption{Caption Caption Caption Caption Caption Caption Caption Caption Caption Caption Caption Caption Caption Caption Caption Caption Caption Caption .}
    \label{fig:uvvis_raster}
\end{figure}

\blindtext
Hier Heatmap und Kratzer zeigen
\blindtext
\begin{figure}
    \centering
    \includegraphics[scale=0.5]{./fig/uvvis_map.png}
    \caption{Caption Caption Caption Caption Caption Caption Caption Caption Caption Caption Caption Caption Caption Caption Caption Caption Caption Caption .}
    \label{fig:uvvis_map}
\end{figure}
\blindtext
\subsection{Probleme Inbetriebnahme}
Mehr Strom/Depodauer nötig (Diskussion), gestaltet sich aber schwierig das umzusetzen.. oxidieren justage strom haut ab blabla
\section{AFM}
Ito lässt sich abbilden, Bild dazu, Aussage über Rauhigkeit treffen, genügend glatt?? \

\blindmathpaper

\subsection{ITO Transmission}

Verschiedene Anzahl Schichten und ggf verschieben
